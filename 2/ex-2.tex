\documentclass[]{article}
\usepackage{lmodern}
\usepackage{amssymb,amsmath}
\usepackage{ifxetex,ifluatex}
\usepackage{fixltx2e} % provides \textsubscript
\ifnum 0\ifxetex 1\fi\ifluatex 1\fi=0 % if pdftex
  \usepackage[T1]{fontenc}
  \usepackage[utf8]{inputenc}
\else % if luatex or xelatex
  \ifxetex
    \usepackage{mathspec}
  \else
    \usepackage{fontspec}
  \fi
  \defaultfontfeatures{Ligatures=TeX,Scale=MatchLowercase}
\fi
% use upquote if available, for straight quotes in verbatim environments
\IfFileExists{upquote.sty}{\usepackage{upquote}}{}
% use microtype if available
\IfFileExists{microtype.sty}{%
\usepackage{microtype}
\UseMicrotypeSet[protrusion]{basicmath} % disable protrusion for tt fonts
}{}
\usepackage[margin=1in]{geometry}
\usepackage{hyperref}
\hypersetup{unicode=true,
            pdftitle={ex-2.R},
            pdfauthor={naelsondouglas},
            pdfborder={0 0 0},
            breaklinks=true}
\urlstyle{same}  % don't use monospace font for urls
\usepackage{color}
\usepackage{fancyvrb}
\newcommand{\VerbBar}{|}
\newcommand{\VERB}{\Verb[commandchars=\\\{\}]}
\DefineVerbatimEnvironment{Highlighting}{Verbatim}{commandchars=\\\{\}}
% Add ',fontsize=\small' for more characters per line
\usepackage{framed}
\definecolor{shadecolor}{RGB}{248,248,248}
\newenvironment{Shaded}{\begin{snugshade}}{\end{snugshade}}
\newcommand{\KeywordTok}[1]{\textcolor[rgb]{0.13,0.29,0.53}{\textbf{#1}}}
\newcommand{\DataTypeTok}[1]{\textcolor[rgb]{0.13,0.29,0.53}{#1}}
\newcommand{\DecValTok}[1]{\textcolor[rgb]{0.00,0.00,0.81}{#1}}
\newcommand{\BaseNTok}[1]{\textcolor[rgb]{0.00,0.00,0.81}{#1}}
\newcommand{\FloatTok}[1]{\textcolor[rgb]{0.00,0.00,0.81}{#1}}
\newcommand{\ConstantTok}[1]{\textcolor[rgb]{0.00,0.00,0.00}{#1}}
\newcommand{\CharTok}[1]{\textcolor[rgb]{0.31,0.60,0.02}{#1}}
\newcommand{\SpecialCharTok}[1]{\textcolor[rgb]{0.00,0.00,0.00}{#1}}
\newcommand{\StringTok}[1]{\textcolor[rgb]{0.31,0.60,0.02}{#1}}
\newcommand{\VerbatimStringTok}[1]{\textcolor[rgb]{0.31,0.60,0.02}{#1}}
\newcommand{\SpecialStringTok}[1]{\textcolor[rgb]{0.31,0.60,0.02}{#1}}
\newcommand{\ImportTok}[1]{#1}
\newcommand{\CommentTok}[1]{\textcolor[rgb]{0.56,0.35,0.01}{\textit{#1}}}
\newcommand{\DocumentationTok}[1]{\textcolor[rgb]{0.56,0.35,0.01}{\textbf{\textit{#1}}}}
\newcommand{\AnnotationTok}[1]{\textcolor[rgb]{0.56,0.35,0.01}{\textbf{\textit{#1}}}}
\newcommand{\CommentVarTok}[1]{\textcolor[rgb]{0.56,0.35,0.01}{\textbf{\textit{#1}}}}
\newcommand{\OtherTok}[1]{\textcolor[rgb]{0.56,0.35,0.01}{#1}}
\newcommand{\FunctionTok}[1]{\textcolor[rgb]{0.00,0.00,0.00}{#1}}
\newcommand{\VariableTok}[1]{\textcolor[rgb]{0.00,0.00,0.00}{#1}}
\newcommand{\ControlFlowTok}[1]{\textcolor[rgb]{0.13,0.29,0.53}{\textbf{#1}}}
\newcommand{\OperatorTok}[1]{\textcolor[rgb]{0.81,0.36,0.00}{\textbf{#1}}}
\newcommand{\BuiltInTok}[1]{#1}
\newcommand{\ExtensionTok}[1]{#1}
\newcommand{\PreprocessorTok}[1]{\textcolor[rgb]{0.56,0.35,0.01}{\textit{#1}}}
\newcommand{\AttributeTok}[1]{\textcolor[rgb]{0.77,0.63,0.00}{#1}}
\newcommand{\RegionMarkerTok}[1]{#1}
\newcommand{\InformationTok}[1]{\textcolor[rgb]{0.56,0.35,0.01}{\textbf{\textit{#1}}}}
\newcommand{\WarningTok}[1]{\textcolor[rgb]{0.56,0.35,0.01}{\textbf{\textit{#1}}}}
\newcommand{\AlertTok}[1]{\textcolor[rgb]{0.94,0.16,0.16}{#1}}
\newcommand{\ErrorTok}[1]{\textcolor[rgb]{0.64,0.00,0.00}{\textbf{#1}}}
\newcommand{\NormalTok}[1]{#1}
\usepackage{graphicx,grffile}
\makeatletter
\def\maxwidth{\ifdim\Gin@nat@width>\linewidth\linewidth\else\Gin@nat@width\fi}
\def\maxheight{\ifdim\Gin@nat@height>\textheight\textheight\else\Gin@nat@height\fi}
\makeatother
% Scale images if necessary, so that they will not overflow the page
% margins by default, and it is still possible to overwrite the defaults
% using explicit options in \includegraphics[width, height, ...]{}
\setkeys{Gin}{width=\maxwidth,height=\maxheight,keepaspectratio}
\IfFileExists{parskip.sty}{%
\usepackage{parskip}
}{% else
\setlength{\parindent}{0pt}
\setlength{\parskip}{6pt plus 2pt minus 1pt}
}
\setlength{\emergencystretch}{3em}  % prevent overfull lines
\providecommand{\tightlist}{%
  \setlength{\itemsep}{0pt}\setlength{\parskip}{0pt}}
\setcounter{secnumdepth}{0}
% Redefines (sub)paragraphs to behave more like sections
\ifx\paragraph\undefined\else
\let\oldparagraph\paragraph
\renewcommand{\paragraph}[1]{\oldparagraph{#1}\mbox{}}
\fi
\ifx\subparagraph\undefined\else
\let\oldsubparagraph\subparagraph
\renewcommand{\subparagraph}[1]{\oldsubparagraph{#1}\mbox{}}
\fi

%%% Use protect on footnotes to avoid problems with footnotes in titles
\let\rmarkdownfootnote\footnote%
\def\footnote{\protect\rmarkdownfootnote}

%%% Change title format to be more compact
\usepackage{titling}

% Create subtitle command for use in maketitle
\newcommand{\subtitle}[1]{
  \posttitle{
    \begin{center}\large#1\end{center}
    }
}

\setlength{\droptitle}{-2em}

  \title{ex-2.R}
    \pretitle{\vspace{\droptitle}\centering\huge}
  \posttitle{\par}
    \author{naelsondouglas}
    \preauthor{\centering\large\emph}
  \postauthor{\par}
      \predate{\centering\large\emph}
  \postdate{\par}
    \date{Thu Jul 26 13:25:08 2018}


\begin{document}
\maketitle

\begin{Shaded}
\begin{Highlighting}[]
\CommentTok{#Organizando o ambiente}
\KeywordTok{setwd}\NormalTok{(}\StringTok{"/home/naelsondouglas/Desktop/Matérias/Estatística/Laboratórios/2"}\NormalTok{)}
\KeywordTok{download.file}\NormalTok{(}\StringTok{"http://www.openintro.org/stat/data/kobe.RData"}\NormalTok{, }\DataTypeTok{destfile =} \StringTok{"kobe.RData"}\NormalTok{)}
\KeywordTok{load}\NormalTok{(}\StringTok{"kobe.RData"}\NormalTok{)}


\CommentTok{#1. Descreva a distribuição das sequências de arremessos. }
\CommentTok{#Qual é o comprimento de sequência típico para o arremessador independente simulado com um percentual de arremesso de 45%?}

\CommentTok{#Sequência típica seria a moda?}

\CommentTok{#Função para calcular a moda. R não tem uma função pre-definida.}
\NormalTok{getmode <-}\StringTok{ }\ControlFlowTok{function}\NormalTok{(v) \{}
\NormalTok{  uniqv <-}\StringTok{ }\KeywordTok{unique}\NormalTok{(v)}
\NormalTok{  uniqv[}\KeywordTok{which.max}\NormalTok{(}\KeywordTok{tabulate}\NormalTok{(}\KeywordTok{match}\NormalTok{(v, uniqv)))]}
\NormalTok{\}}

\KeywordTok{getmode}\NormalTok{(}\KeywordTok{calc_streak}\NormalTok{(sim_basket))}
\end{Highlighting}
\end{Shaded}

\begin{verbatim}
## [1] 0
\end{verbatim}

\begin{Shaded}
\begin{Highlighting}[]
\CommentTok{#Dá zero e fica sem graça. Vamos remover as sequências zeradas do espaço para ver algum número}
\KeywordTok{getmode}\NormalTok{(seqs_sim[seqs_sim}\OperatorTok{!=}\DecValTok{0}\NormalTok{])}
\end{Highlighting}
\end{Shaded}

\begin{verbatim}
## [1] 1
\end{verbatim}

\begin{Shaded}
\begin{Highlighting}[]
\CommentTok{#Quão longa é a sequência mais longa de cestas em 133 arremessos?}
\KeywordTok{max}\NormalTok{(seqs_sim)}
\end{Highlighting}
\end{Shaded}

\begin{verbatim}
## [1] 5
\end{verbatim}

\begin{Shaded}
\begin{Highlighting}[]
\CommentTok{#02 Se você rodasse a simulação do arremessador independente uma segunda vez, como você acha que}
\CommentTok{# seria a distribuição de sequências em relação à distribuição da questão acima? Exatamente a mesma?}
\CommentTok{#   Mais ou menos parecida? Completamente diferente? Explique seu raciocínio.}

\KeywordTok{print}\NormalTok{(}\StringTok{"Muito parecidas. Nâo idênticas, porém parecidas. Os espaços amostrais teriam a mesma dimensão e seriam simulados com as mesmas probabilidades, logo seriam muito semelhantes"}\NormalTok{)}
\end{Highlighting}
\end{Shaded}

\begin{verbatim}
## [1] "Muito parecidas. Nâo idênticas, porém parecidas. Os espaços amostrais teriam a mesma dimensão e seriam simulados com as mesmas probabilidades, logo seriam muito semelhantes"
\end{verbatim}

\begin{Shaded}
\begin{Highlighting}[]
\CommentTok{# 3. Como a distribuição dos comprimentos de sequência de Kobe Bryant, analisada na página 2, se}
\CommentTok{# comparam à distribuição de comprimentos de sequência do arremessador simulado? Utilizando}
\CommentTok{# essa comparação, você tem evidência de que o modelo das mãos quentes se ajusta aos padrões de}
\CommentTok{# arremessos de Kobe? Explique.}

\KeywordTok{print}\NormalTok{(}\StringTok{"Olhando os sumários e os boxplots (ou apenas o boxplot), podemos ver que os arremessos do Kobe são sumarizados de forma}
\StringTok{semelhante ao modelo de eventos independentes. Só os outliers(Max) que são um pouco diferentes, mas como o nome já diz: eles são outliers. }
\StringTok{Os quartis dentro dos boxplots se encaixam perfeitamente"}\NormalTok{)}
\end{Highlighting}
\end{Shaded}

\begin{verbatim}
## [1] "Olhando os sumários e os boxplots (ou apenas o boxplot), podemos ver que os arremessos do Kobe são sumarizados de forma\nsemelhante ao modelo de eventos independentes. Só os outliers(Max) que são um pouco diferentes, mas como o nome já diz: eles são outliers. \nOs quartis dentro dos boxplots se encaixam perfeitamente"
\end{verbatim}

\begin{Shaded}
\begin{Highlighting}[]
\KeywordTok{summary}\NormalTok{(seqs_sim)}
\end{Highlighting}
\end{Shaded}

\begin{verbatim}
##    Min. 1st Qu.  Median    Mean 3rd Qu.    Max. 
##  0.0000  0.0000  0.0000  0.6145  1.0000  5.0000
\end{verbatim}

\begin{Shaded}
\begin{Highlighting}[]
\KeywordTok{summary}\NormalTok{(seqs_kobe)}
\end{Highlighting}
\end{Shaded}

\begin{verbatim}
##    Min. 1st Qu.  Median    Mean 3rd Qu.    Max. 
##  0.0000  0.0000  0.0000  0.7632  1.0000  4.0000
\end{verbatim}

\begin{Shaded}
\begin{Highlighting}[]
\KeywordTok{boxplot}\NormalTok{(seqs_sim,seqs_kobe)}
\end{Highlighting}
\end{Shaded}

\includegraphics{ex-2_files/figure-latex/unnamed-chunk-1-1.pdf}

\begin{Shaded}
\begin{Highlighting}[]
\CommentTok{# 4. Quais conceitos do livro são abordados neste laboratório? Quais conceitos, se houver algum, que}
\CommentTok{# não são abordados no livro? Você viu esses conceito em algum outro lugar, p.e., aulas, seções de}
\CommentTok{# discussão, laboratórios anteriores, ou tarefas de casa? Seja específico em sua resposta.}
\KeywordTok{print}\NormalTok{(}\StringTok{"Questão pedindo feedback do livro Openintro. Eu estudei pelo livro do Marcos Magalhães, então não tem como dar feedback do openintro"}\NormalTok{)}
\end{Highlighting}
\end{Shaded}

\begin{verbatim}
## [1] "Questão pedindo feedback do livro Openintro. Eu estudei pelo livro do Marcos Magalhães, então não tem como dar feedback do openintro"
\end{verbatim}

\begin{Shaded}
\begin{Highlighting}[]
\CommentTok{#Plotando o script em .PDF}
\CommentTok{#Essa linha fica comentada porque ela ao ser chamada executa o script, mas ela mesma tá escrita no script. Isso causaria um loop eterno}
\CommentTok{#Chamo ela pelo console, deixando aqui comentada mesmo só para registro}
\CommentTok{#As vezes dá erro na hora de baixar o dataset, é só tentar de novo.}
\CommentTok{#rmarkdown::render("ex-2.R")}
\end{Highlighting}
\end{Shaded}


\end{document}
